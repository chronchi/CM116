\documentclass[a4paper,10pt]{article}

\usepackage[utf8]{inputenc}
\usepackage[T1]{fontenc}
\usepackage[brazil]{babel}
\usepackage{amssymb,amsfonts,amsmath}
\usepackage{amsthm}
%\usepackage{dsfont}
\usepackage{enumerate}
\usepackage{graphicx}
\usepackage{multirow}
\usepackage{xcolor}
\usepackage{latexsym}
\usepackage{makeidx}
\usepackage{setspace}
\usepackage[matrix,arrow,arc,all]{xy}
\usepackage{wrapfig}
\usepackage{mathtools}
\usepackage{booktabs}


%\usepackage[brazil,portuges]{babel}
%\usepackage[ansinew]{inputenc}
%\usepackage{graphics,graphicx}
%\renewcommand{\baselinestretch}{1.2}
\usepackage{geometry}
\usepackage{amssymb}
\usepackage{multicol}
\usepackage[normalem]{ulem} % sublinhado
\usepackage{verbatim} % para os comenterios em blocos \begin{comment}
\geometry{top=30mm,bottom=30mm,left=30mm,right=30mm}
\usepackage[pdftex]{hyperref}
\hypersetup{
    colorlinks=true,               % false: boxed links; true: colored links
        linkcolor=black,            % color of internal links
        citecolor=blue,            % color of links to bibliography
%        filecolor=magenta,         % color of file links
    urlcolor=blue,
    }
\usepackage[matrix,arrow,arc,all]{xy}
\usepackage[lastexercise]{exercise}


%--------------------------------------------------------------------------------------------


%--------------------------------------------------------------------------------------------

\usepackage{algorithm}
\usepackage{algpseudocode}
\makeatletter
\renewcommand{\ALG@name}{Algoritmo}
\renewcommand{\listalgorithmname}{Lista de algoritmos}
\renewcommand{\algorithmicwhile}{\textbf{enquanto}}
\renewcommand{\algorithmicdo}{\textbf{faça}}
\renewcommand{\algorithmicend}{\textbf{fim}}
\renewcommand{\algorithmicfor}{\textbf{para}}

%-------------------------------------------------------------------------------------------

\usepackage{amstext}
%\usepackage{wrapfig}
\usepackage{multicol,colordvi,epsfig,float,color}
\usepackage{tikz}
\usepackage[matrix,arrow,arc,all]{xy}
\usepackage{textcomp}
\usepackage{indentfirst}
\usepackage{booktabs}
\usetikzlibrary{matrix,decorations.pathreplacing,calc}

%--------------------------------------------------------------------------------------------

%\setlength{\topmargin}{-0.4in}\setlength{\topskip}{0.3in}    % between header and text
%\setlength{\textheight}{9.5in} % height of main text
%\setlength{\textwidth}{6.25in}    % width of text
%\setlength{\oddsidemargin}{0.2in} % odd page left margin
%\setlength{\evensidemargin}{0.2in} % even page left margin
%\setlength{\parindent}{1cm}

%--------------------------------------------------------------------------------------------

%\newenvironment{AMS}{\small\it AMS subject classification:}{}
%\newenvironment{keywords}{\small\it Key words:}{}
%\newenvironment{proof}{Proof. }

\newenvironment{solution}
  {\begin{proof}[Solução]}
  {\end{proof}}
\newtheorem{theorem}{\sc Teorema}[section]
\newtheorem{teo}[theorem]{\scshape Teorema}
\newtheorem{lema}[theorem]{\scshape Lema}
\newtheorem{cor}[theorem]{\scshape Corol\'ario}
\newtheorem{propo}[theorem]{\scshape Proposi\c{c}\~ao}
\newtheorem{definicao}{\scshape Defini\c{c}\~ao}[section]
\newtheorem{ex}{\scshape Exemplo}[section]
\newtheorem{obs}{Observação}[section]
\newcommand{\norm}[1]{\left\lVert#1\right\rVert}
\DeclareMathOperator{\Ker}{Ker}
\usepackage{enumitem}
\setenumerate[0]{label=(\alph*)}
\DeclareMathOperator{\posto}{posto}
\newcommand{\im}[1]{\textnormal{Im}(#1)}
\newcommand{\spann}[1]{\textnormal{span}\{#1\}}
\DeclarePairedDelimiter\ceil{\lceil}{\rceil}
\DeclarePairedDelimiter\floor{\lfloor}{\rfloor}
\DeclareMathOperator{\argmin}{argmin}


%matrix environment for side brackets
\pgfkeys{tikz/mymatrixenv/.style={decoration=brace,every left delimiter/.style={xshift=3pt},every right delimiter/.style={xshift=-3pt}}}
\pgfkeys{tikz/mymatrix/.style={matrix of math nodes,left delimiter=[,right delimiter={]},inner sep=2pt,column sep=1em,row sep=0.5em,nodes={inner sep=0pt}}}
\pgfkeys{tikz/mymatrixbrace/.style={decorate,thick}}
\newcommand\mymatrixbraceoffseth{0.5em}
\newcommand\mymatrixbraceoffsetv{0.2em}

% Now the commands to produce the braces. (I'll explain below how to use them.)
\newcommand*\mymatrixbraceright[4][m]{
    \draw[mymatrixbrace] ($(#1.north west)!(#1-#3-1.south west)!(#1.south west)-(\mymatrixbraceoffseth,0)$)
        -- node[left=2pt] {#4}
        ($(#1.north west)!(#1-#2-1.north west)!(#1.south west)-(\mymatrixbraceoffseth,0)$);
}
\newcommand*\mymatrixbraceleft[4][m]{
    \draw[mymatrixbrace] ($(#1.north east)!(#1-#2-1.north east)!(#1.south east)+(\mymatrixbraceoffseth,0)$)
        -- node[right=2pt] {#4}
        ($(#1.north east)!(#1-#3-1.south east)!(#1.south east)+(\mymatrixbraceoffseth,0)$);
}
\newcommand*\mymatrixbracetop[4][m]{
    \draw[mymatrixbrace] ($(#1.north west)!(#1-1-#2.north west)!(#1.north east)+(0,\mymatrixbraceoffsetv)$)
        -- node[above=2pt] {#4}
        ($(#1.north west)!(#1-1-#3.north east)!(#1.north east)+(0,\mymatrixbraceoffsetv)$);
}
\newcommand*\mymatrixbracebottom[4][m]{
    \draw[mymatrixbrace] ($(#1.south west)!(#1-1-#3.south east)!(#1.south east)-(0,\mymatrixbraceoffsetv)$)
        -- node[below=2pt] {#4}
        ($(#1.south west)!(#1-1-#2.south west)!(#1.south east)-(0,\mymatrixbraceoffsetv)$);
}

\begin{document}

  \begin{titlepage}
   \vfill
      \begin{center}
          {\large \textbf{UNIVERSIDADE FEDERAL DO PARANÁ}} \\[2.5cm]

          {\large \textbf{Carlos Ronchi}}\\[4cm]


          {\Large Lista 3 - CM116}\\[4cm]

          \hspace{.45\textwidth} %posiciona a minipage
          \begin{minipage}{.5\textwidth}
              \large Lista de exercícios para a disciplina de tópicos em matemática aplicada I.\\[1cm]
              Professor Dr. Geovani Grapiglia.
          \end{minipage}
          \vfill

          \vspace{2cm}

          \large \textbf{Curitiba}

          \large \textbf{Junho de 2017}
      \end{center}
  \end{titlepage}

  \begin{solution}{(Exercício 1)}

    Para este exercício vamos usar os seguintes lemas.

    \begin{lema}\label{lem:1}
      Seja $A \in \mathbb{R}^{m\times n}$, então $\Ker(A^TA) = \Ker(A)$.
    \end{lema}
    \begin{proof}

        Seja $x\in \Ker(A)$, então $A^TAx = A^T0 = 0$. Portanto, $x\in \Ker(A^TA)$.

        Agora, seja $x \in \Ker(A^TA)$, então $A^TAx = 0$. Multiplicando por $x^T$ em
        ambos os lados, temos que
        \begin{equation*}
          \norm{Ax} = 0.
        \end{equation*}
        Portanto, $Ax = 0$ e assim $x \in \Ker(A)$. Concluimos então que $\Ker(A^TA) = \Ker(A)$.
    \end{proof}

    \begin{lema}\label{lem:2}
      Seja $A \in \mathbb{R}^{m \times n}$, então $A^TA$ possui autovalores não-negativos.
    \end{lema}
    \begin{proof}
      Seja $v$ um autovetor associado a $\lambda$ da matriz $A^TA$, então
      \begin{equation*}
        A^TAv = \lambda v \rightarrow v^TA^TAv = \lambda v^Tv.
      \end{equation*}
      Como $A^TA$ é uma matriz semi definida positiva, temos que
      \begin{equation*}
        \lambda v^Tv \geq 0.
      \end{equation*}
      Além disso, como $v$ é um autovetor, $v^Tv > 0$, portanto,
      \begin{equation*}
        \lambda \geq 0.
      \end{equation*}
    \end{proof}

    Vamos agora provar o SVD. Seja $A\in \mathbb{R}^{m\times n}$ uma matriz qualquer e suponha
    que $m \geq n$. Seja a matriz $A^TA \in \mathbb{R}^{n \times n}$ e
    sejam $x_1, \dots, x_n$ autovetores ortonormais e $\lambda_1,\dots,\lambda_n$ os seus
    respectivos autovalores. Vamos ordenar os autovalores de forma que
    $\lambda_1 \geq \dots \lambda_n$. E seja $r$ tal que
    \begin{equation*}
      \lambda_1 \geq \dots \geq \lambda_r > \lambda_{r+1} = \dots = \lambda_n.
    \end{equation*}
    Defina agora os vetores
    \begin{equation*}
      q_i = \frac{1}{\sqrt{\lambda_i}}Ax_i, \quad \textnormal{para $i\leq r$},
    \end{equation*}
    e chame de $\sigma_i = \sqrt{\lambda_i}$, para todo $i=1,\dots,n$.

    Observe que $q_i^Tq_j = 0$ para todo $i \neq j$ e $q_i^Tq_i = 1$. Portanto, vamos
    completar $\{q_1,\dots,q_r\}$ de forma a obter uma base $\{q_1,\dots,q_r,q_{r+1},\dots,q_m\}$
    de $\mathbb{R}^m$. Sejam agora as matrizes

    \begin{equation*}
      U = \begin{pmatrix}
          \vert & & \vert \\
          q_1 & \dots & q_m \\
          \vert & & \vert
        \end{pmatrix},
        \quad \quad
      V = \begin{pmatrix}
              \vert & & \vert \\
              x_1 & \dots & x_n \\
              \vert & & \vert
            \end{pmatrix}.
    \end{equation*}

    Note que $x_j \in \Ker(A^TA)$ para $j > r$, pois o autovalor associado a $x_j$ é $0$. Pelo
    Lema \ref{lem:1}, $Ax_j = 0$. Portanto,
    \begin{equation*}
      (U^TAV)_{ij} = q_i^TAx_j =
      \left\{
        \begin{split}
          &\frac{1}{\sigma_i} x_iA^TAx_j && \textnormal{se } j \leq r \\
          &0 && \textnormal{se } j > r
        \end{split}
      \right.
      =
      \left\{
        \begin{split}
          &\sigma_i\delta_{ij} && \textnormal{se } j \leq r \\
          &0 && \textnormal{se } j > r
        \end{split}
      \right.
      ,
    \end{equation*}

    pois
    \begin{equation*}
      \frac{1}{\sigma_i}x_i^TA^TAx_j = \frac{\lambda_j}{\sigma_i}x_i^Tx_j = \sigma_i\delta_{ij}.
    \end{equation*}
    Portanto, $U^TAV$ é dada por
    \begin{equation*}
      \begin{tikzpicture}[mymatrixenv]
        \matrix[mymatrix] (m)  {
            \sigma_1 & 0 & \cdots & 0 \\
            0 & \sigma_2 & \cdots & 0 \\
            \vdots & \vdots & \ddots & \vdots \\
            0 & 0 & \cdots & \sigma_n \\
            0 & 0 & \cdots & 0 \\
            0 & 0 & \cdots & 0 \\
        };
        \mymatrixbraceright{1}{4}{n}
        \mymatrixbracetop{1}{4}{n}
        \mymatrixbraceleft{5}{6}{m-n}
      \end{tikzpicture}
    \end{equation*}

    Denotando por $\Sigma$ a matriz acima, temos que

    \begin{equation*}
      A = U\Sigma V^T
    \end{equation*}

    Seja agora $m<n$. Considere a matriz simétrica $AA^T$. Do memso modo que a anrerior,
    temos que seus autovalores são todos não negativos. Sejam $v_1,\dots,v_m$ os autovalores
    ortonormais associados a $\lambda_1, \dots, \lambda_m$ autovalores. Ordene
    $\lambda_1 \geq \dots \geq \lambda_m$. Denote por $\sigma_i = \sqrt{\lambda_i}$ para
    todo $i=1,\dots,m$. Seja $r$ tal que
    \begin{equation*}
      \lambda_1 \geq \dots \geq \lambda_r > \lambda_{r+1} = \dots = \lambda_m.
    \end{equation*}
    Para $i=1,\dots,r$, denote por
    \begin{equation*}
      q_i = \frac{1}{\sigma_i}A^Tv_i.
    \end{equation*}
    Note que os $q_i$'s são ortonormais. Completando a base e utilizando o processo
    de ortogonalização de Gram-Schimdt, temos uma base $\{q_1,\dots,q_r,q_{r+1}, \dots, q_n\}$
    ortornormal de $\mathbb{R}^n$. Sejam agora as matrizes

    \begin{equation*}
      U = \begin{pmatrix}
          \vert & & \vert \\
          v_1 & \dots & v_m \\
          \vert & & \vert
        \end{pmatrix},
        \quad \quad
      V = \begin{pmatrix}
              \vert & & \vert \\
              q_1 & \dots & q_n \\
              \vert & & \vert
            \end{pmatrix}.
    \end{equation*}

    Além disso, de modo análogo ao Lema \ref{lem:1}, temos que $\Ker(AA^T) = \Ker(A^T)$,
    basta substituir $A$ por $A^T$. Portanto,

    \begin{equation*}
      (U^TAV)_{ij} = v_i^TAq_j =
      \left\{
        \begin{split}
          &\frac{1}{\sigma_i} v_iAA^Tv_j && \textnormal{se } j \leq r \\
          &0 && \textnormal{se } j > r
        \end{split}
      \right.
      =
      \left\{
        \begin{split}
          &\sigma_j\delta_{ij} && \textnormal{se } j \leq r \\
          &0 && \textnormal{se } j > r
        \end{split}
      \right.
      ,
    \end{equation*}

    pois $v_i \in \Ker(A^T)$ e
    \begin{equation*}
      \frac{1}{\sigma_j}v_i^TAA^Tv_j = \frac{\lambda_i}{\sigma_j}v_i^Tv_j = \sigma_j\delta_{ij}.
    \end{equation*}

    Portanto, a matriz $U^TAV$ é dada por

    \begin{equation*}
      \begin{tikzpicture}[mymatrixenv]
        \matrix[mymatrix] (m)  {
            \sigma_1 & 0 & \cdots & 0 & 0 & \cdots & 0\\
            0 & \sigma_2 & \cdots & 0 & 0 & \cdots & 0\\
            \vdots & \vdots & \ddots & \vdots & 0 & \cdots & 0\\
            0 & 0 & \cdots & \sigma_m & 0 & \cdots & 0\\
        };
        \mymatrixbraceright{1}{4}{m}
        \mymatrixbracetop{1}{4}{m}
        \mymatrixbracetop{5}{7}{n-m}
      \end{tikzpicture}.
    \end{equation*}


  \end{solution}


  \begin{solution}{(Exercício 2)}

    Suponha $m<n$, então
    \begin{equation*}
      \begin{split}
      Ax &= U\Sigma V^T = \begin{pmatrix}
          \vert & & \vert \\
          u_1 & \dots & u_m \\
          \vert & & \vert
        \end{pmatrix}
        \begin{pmatrix}
          \sigma_1 & & &  &0 \\
          &\sigma_2 & & & 0 \\
          & & \ddots & & 0 \\
          & & &\sigma_m & 0
        \end{pmatrix}
        \begin{pmatrix}
            \-- & v_1^T & \-- \\
             & \vdots &  \\
            \-- & v_n^T & \--
        \end{pmatrix}
        \begin{pmatrix}
          x_1 \\
          x_2 \\
          \vdots \\
          x_n
        \end{pmatrix} \\
        &= \begin{pmatrix}
            \vert & & \vert \\
            u_1 & \dots & u_m \\
            \vert & & \vert
          \end{pmatrix}
          \begin{pmatrix}
            \sigma_1v_1^Tx \\
            \vdots \\
            \sigma_mv_m^Tx
          \end{pmatrix}
      \end{split}
    \end{equation*}

    Seja agora $x = v_i$, então
    \begin{equation*}
      \begin{split}
        Av_i &= \begin{pmatrix}
            \vert & & \vert \\
            u_1 & \dots & u_m \\
            \vert & & \vert
          \end{pmatrix}
          \begin{pmatrix}
            \sigma_1v_1^Tv_i \\
            \vdots \\
            \sigma_mv_m^Tv_i
          \end{pmatrix}  \\
          &= \begin{pmatrix}
              \vert & & \vert \\
              u_1 & \dots & u_m \\
              \vert & & \vert
            \end{pmatrix}
            \begin{pmatrix}
              0 \\
              0 \\
              \vdots \\
              \sigma_i \\
              0 \\
              \vdots \\
              0
            \end{pmatrix} \\
          &= \begin{pmatrix}
            \sigma_iu_{i1} \\
            \sigma_iu_{i2} \\
            \vdots \\
            \sigma_iu{ii} \\
            \vdots \\
            \sigma_iu_{im}
          \end{pmatrix} = \sigma_iu_i
      \end{split}
    \end{equation*}

    Já para $A^Tu_i$, temos que
    \begin{equation*}
      \begin{split}
      A^Tx &= V\Sigma^TU^T =
        \begin{pmatrix}
          \vert & & \vert \\
          v_1 & \dots & v_n \\
          \vert & & \vert
        \end{pmatrix}
        \begin{pmatrix}
          \sigma_1 & & &  \\
          &\sigma_2 & &  \\
          & & \ddots &  \\
          & & &\sigma_m \\
          0 & \dots & 0 & 0
        \end{pmatrix}
        \begin{pmatrix}
            \-- & u_1^T & \-- \\
             & \vdots &  \\
            \-- & u_m^T & \--
        \end{pmatrix}
        \begin{pmatrix}
          x_1 \\
          \vdots \\
          x_m
        \end{pmatrix} \\
        &= \begin{pmatrix}
            \vert & & \vert \\
            v_1 & \dots & v_n \\
            \vert & & \vert
          \end{pmatrix}
          \begin{pmatrix}
          \sigma_1u_1^Tx \\
          \vdots \\
          \sigma_mu_m^Tx \\
          0 \\
          \vdots \\
          0
        \end{pmatrix}
      \end{split}
    \end{equation*}
    Portanto, para $x=u_i$
    \begin{equation*}
      A^Tu_i = \sigma_i v_i.
    \end{equation*}

    Para calcular os valores de $\sigma_i$, $u_i$ e $v_i$,
    \begin{equation*}
      A^TAv_i = \sigma_iA^Tu_i = \sigma_i^2v_i.
    \end{equation*}
    Logo, $v_i$ é o autovetor de $A^TA$ associado ao autovalor $\sigma_i^2$ da matriz
    $A^TA$. E
    \begin{equation*}
      AA^Tu_i = \sigma_iAv_i = \sigma_i^2u_i.
    \end{equation*}
    Logo, $u_i$ é o autovetor de $AA^T$ associado ao autovalor $\sigma_i^2$ da matriz
    $AA^T$. Isto pode ser visto também através da demonstração do SVD pelo exercício 1.
  \end{solution}


  \begin{solution}{(Exercício 3)}

    \begin{enumerate}
      \item De fato, como $A = U\Sigma V^T$, e como $U$, $V^T$ são matrizes ortogonais,
      temos que $\posto( U\Sigma V^T) = \posto(\Sigma)$. E como $\posto(\Sigma) = r$,
      já que
      \begin{equation*}
        \Sigma =
        \begin{pmatrix}
          \sigma_1 &          &       &          & \\
                   & \sigma_2 &       &          & \\
                   &          & \ddots&          & \\
                   &          &       & \sigma_r & \\
                   &          &       &          & 0 & \\
                   & & & & & \ddots& \\
                   & & & & & & 0 \\
                   & & & \vdots & & & \\
                   0 & & & & & & 0
        \end{pmatrix}
      \end{equation*}
      \begin{equation*}
        \Rightarrow \posto(A) = \posto(\Sigma) = r.
      \end{equation*}

      \item Seja $x \in \mathbb{R}^n$, então
        \begin{equation*}
          x = b_1v_1 + \dots b_nv_n,
        \end{equation*}
        onde $v_1,\dots,v_n$ são os vetores singulares à direita da decomposição
        SVD da matriz A. Então, pelo exercício 2
        \begin{equation*}
          \begin{split}
            &Ax = A(b_1v_1 + \dots b_nv_n) = b_1\sigma_1u_1 + \dots b_r\sigma_ru_n \\
            & \Rightarrow \im{A} \in \textnormal{span}\{u_1,\dots,u_r\}
          \end{split}
        \end{equation*}

        Seja $x\in \textnormal{span}\{u_1,\dots,u_r\}$. Pelo exercício 2, $u_i =
        \frac{1}{\sigma_i}Av_i$, portanto, $x \in \im{A}$.
        \begin{equation*}
          \im{A} = \textnormal{span}\{u_1,\dots,u_r\}.
        \end{equation*}

      \item Seja $x \in \Ker(A)$. Como $x\in \mathbb{R}^n$, $x = b_1v_1 + \dots b_nv_n$.
        \begin{equation*}
            A(b_1v_1 + \dots + b_nv_n) = 0 \Rightarrow b_1\sigma_1 u_1 + \dots + b_r\sigma_r u_r = 0
        \end{equation*}

        Como os vetores $u_1, \dots, u_m$ formam uma base para $\mathbb{R}^m$ e $\sigma_1,\dots,\sigma_r
        \neq 0$, temos que
        \begin{equation*}
          b_1 = \dots = b_r.
        \end{equation*}

        Portanto, $x \in \spann{v_{r+1}, \dots, v_n}$. Do mesmo modo, temos que
        $Av_i = \sigma_i u_i$. Se $i>r$, então $\sigma_i = 0$.
        \begin{equation*}
          Av_i = 0 \Rightarrow v_i \in \Ker(A) \Rightarrow \Ker(A) = \spann{v_{r+1}, \dots, v_n}.
        \end{equation*}

        \item De fato, note que pelos exercícios anteriores, se $m<n$, então
          \begin{equation*}
            \begin{split}
              A = U\Sigma V^T &=
              \begin{pmatrix}
                  \vert & & \vert \\
                  u_1 & \dots & u_m \\
                  \vert & & \vert
                \end{pmatrix}
                \begin{pmatrix}
                  \sigma_1 & & &  &0 \\
                  &\sigma_2 & & & 0 \\
                  & & \ddots & & 0 \\
                  & & &\sigma_m & 0
                \end{pmatrix}
                \begin{pmatrix}
                    \-- & v_1^T & \-- \\
                     & \vdots &  \\
                    \-- & v_n^T & \--
                \end{pmatrix} \\
                &= \begin{pmatrix}
                    \vert & & \vert \\
                    u_1 & \dots & u_m \\
                    \vert & & \vert
                  \end{pmatrix}
                  \begin{pmatrix}
                    \sigma_1v_1^Tx \\
                    \vdots \\
                    \sigma_mv_m^Tx
                  \end{pmatrix} \\
                &= \sum_{i=1}^{m}\sigma_iu_iv_i^T = \sum_{i=1}^{r}\sigma_iu_iv_i^T
            \end{split}
          \end{equation*}
    \end{enumerate}
  \end{solution}

  \begin{solution}{(Exercício 4)}

    \begin{enumerate}
      \item Esta é a norma de Froebenius de uma matriz A, dada por
        \begin{equation*}
          \norm{A}_F^2 = tr(AA^T),
        \end{equation*}
        onde $tr$ é o operador traço. Além disso, se $U$ for uma matriz ortogonal, então
        \begin{equation*}
          \norm{UA}_F^2 = tr(UAA^TU^T) = tr(AA^TU^TU) = tr(AA^T) = \norm{A}_F^2.
        \end{equation*}
        Ainda mais, isto vale para multiplicação à direita, ou seja
        \begin{equation*}
          \norm{AU}_F^2 = tr(AUU^TA^T) = tr(AA^T) = \norm{A}_F^2.
        \end{equation*}
        Portanto,
        \begin{equation*}
          \norm{A}_F^2 = \norm{U\Sigma V^T}_F^2 = \norm{\Sigma}_F^2 = \sigma_1 + \dots + \sigma_p,
        \end{equation*}
        $p = \min\{m,n\}$.

      \item Note que se $x$ é um vetor cuja norma é igual a um e $V$ é uma matriz ortogonal, então
      $\norm{Vx} = \norm{x} = 1$. Portanto,
      \begin{equation*}
        \begin{split}
          \norm{A}_2 &= \sup_{\norm{x} = 1}\norm{Ax} = \sup_{\norm{x} = 1} \norm{U\Sigma V^Tx} \\
          &= \sup_{\norm{x} = 1} \norm{\Sigma V^Tx} = \sup_{\norm{V^Tx} = }\norm{\Sigma V^Tx}
          = \sup_{\norm{y} = 1}\norm{\Sigma y} \\
          &= \norm{\Sigma}_2
        \end{split}
      \end{equation*}

      Suponha que $m<n$, então
      \begin{equation*}
        \Sigma = \begin{pmatrix}
          \sigma_1 & & &  &0 \\
          &\sigma_2 & & & 0 \\
          & & \ddots & & 0 \\
          & & &\sigma_m & 0
        \end{pmatrix}
      \end{equation*}

      Logo,

      \begin{equation*}
        \norm{\Sigma}_2 = \sup_{\norm{x} = 1} \norm{\Sigma x} = \sup_{\norm{x} = 1}
        \left|\sqrt{\sigma_1^2x_1^2 + \dots + \sigma_m^2x_m^2}\right| \leq \sup_{\norm{x} = 1} \sigma_1 \norm{x} = \sigma_1,
      \end{equation*}
      já que $\sigma_1 > 0$.
      Agora seja $x = (1, 0, \dots, 0)\in \mathbb{R}^n$. Então, $\norm{x}_2 = 1$.
      Portanto, $\norm{\Sigma x}_2 = \sigma_1$. Ou seja, $\norm{\Sigma}_2 \geq \sigma_1$.
      \begin{equation*}
        \norm{A}_2 = \norm{\Sigma}_F = \sigma_1.
      \end{equation*}

      de modo análogo temos que para $m \geq n$.

      \item Seja $m\geq n$. Então,

      \begin{equation*}
        \Sigma = \begin{pmatrix}
          \sigma_1 & & &\\
          &\sigma_2 & & \\
          & & \ddots &  \\
          & & &\sigma_n \\
          0 & 0 & \dots & 0
        \end{pmatrix}
      \end{equation*}

      Assim,
      \begin{equation*}
        \begin{split}
          \min_{x\neq 0} \frac{\norm{Ax}}{\norm{x}} &= \min_{x\neq 0}{\norm{\Sigma x}}{\norm{x}} \\
          &= \min_{x\neq 0} \frac{\left|\sqrt{\sigma_1^2x_1^2 + \dots + \sigma_n^2x_n^2}\right|}{\norm{x}} \\
          &\geq \sigma_n \min_{x\neq 0} \frac{\norm{x}}{\norm{x}} = \sigma_n.
        \end{split}
      \end{equation*}

      Por outro lado, seja $x = (0, 0, \dots, 0, 1) \in \mathbb{R}^m$, então
      \begin{equation*}
        \min_{y\neq 0} \frac{\norm{Ay}_2}{\norm{y}_2} \leq \frac{\norm{\Sigma x}_2}{\norm{x}_2} = \sigma_n.
      \end{equation*}

      \begin{equation*}
        \Rightarrow \min_{x\neq 0} \frac{\norm{Ax}_2}{\norm{x}_2} = \sigma_n.
      \end{equation*}
    \end{enumerate}
  \end{solution}


  \begin{solution}{(Exercício 5)}

    Note que $\norm{A - A_k}_2 = \sigma_{k+1}$, pois
    \begin{equation*}
      \norm{A - A_k}_2 = \norm{\sum_{j=k+1}^{n}\sigma_ju_jv_J^T} = \sigma_{k+1}
    \end{equation*}
    pelo exercício 4-b. Portanto, resta mostrar que
    \begin{equation*}
        \min_{\posto(B) = k} \norm{A-B}_2 = \norm{A-A_k}.
    \end{equation*}

    Como $posto(A_k) = k$, já que $k<r$, temos que
    \begin{equation*}
      \min_{\posto(B) = k} \norm{A-B}_2 \leq \norm{A-A_k}.
    \end{equation*}

    Resta mostrar que
    \begin{equation*}
      \min_{\posto(B) = k} \norm{A-B}_2 \geq \norm{A-A_k}.
    \end{equation*}

    Seja $U\Sigma V^T$ a decomposição SVD da matriz A. Seja $B$ uma matriz de posto
    igual a $k$. Como $k<r$, temos pelo teorema do núcleo e da imagem que
    dim$(\Ker(B)) = n-k$. Além disso, restrinja a matriz V para $V_k$,
    sendo $V_k$ a matriz das primeiras $k+1$  colunas
    de V, portanto, $ dim(V_k) = k+1 $. Logo,
    \begin{equation*}
      dim(\Ker(B)) + dim(V_k) = n-k + k + 1 = n+1.
    \end{equation*}

    Como a soma das dimensões é maior que $n$ e ambos espaços estão contidos em
    $\mathbb{R}^n$, temos que
    \begin{equation*}
      \Ker(B) \cap V_k \neq \emptyset.
    \end{equation*}

    Tome $x \in \Ker(B) \cap V_k$, tal que $\norm{x}_2 = 1$. Então

    \begin{equation*}
      \norm{A-B}_2 \geq \norm{Ax-Bx}_2 = \norm{A_{k+1}x}_2.
    \end{equation*}

    Como $x \in V_k$, $x = c_1v_1 + \dots + c_{k+1}v_{k+1}$, então

    \begin{equation*}
      \begin{split}
        A_{k+1}x &= \sum_{i=1}^{k+1}\sigma_iu_iv_i^Tx = \sum_{i=1}^{k+1}\sigma_iu_iv_i^T
      (c_1v_1 + \dots + c_{k+1}v_{k+1}) \\
        &= \sum_{i=1}^{k+1}\sigma_ic_iu_i.
      \end{split}
    \end{equation*}

    Utilizando o resultado acima, temos que
    \begin{equation*}
      \norm{A_{k+1}x}_2^2 = \norm{\sum_{i=1}^{k+1}\sigma_ic_iu_i}_2 \geq
      \sigma_{k+1} \norm{\sum_{i=1}^{k+1}c_iu_i}_2.
    \end{equation*}

    E note que
    \begin{equation*}
      \begin{split}
        1 = \norm{x}_2^2 = \left\langle \sum_i c_iv_i, \sum_j c_jv_j \right\rangle =
        \sum_{i=1}^{k+1}c_i^2.
      \end{split}
    \end{equation*}

    Mas como os vetores $u_1,\dots,u_{k+1}$ são ortonormais, temos que
    \begin{equation*}
      \norm{\sum_{i=1}^{k+1}c_iu_i}_2^2 = \left\langle \sum_i c_iu_i, \sum_j c_ju_j \right\rangle
      = \sum_{i=1}^{k+1}c_i^2.
    \end{equation*}

    Portanto, juntando os resultados acima, temos que
    \begin{equation*}
      \norm{A-B}_2 \geq \sigma_{k+1}
    \end{equation*}

    Assim, obtemos o resultado desejado.
    \begin{equation*}
        \min_{\posto(B) = k} \norm{A-B}_2 = \norm{A - A_k}_2 = \sigma_{k+1}.
    \end{equation*}
  \end{solution}

  \begin{solution}{(Exercício 6)}

    Considere a SVD de $A\in \mathbb{R}^{m\times n}$ dada no Exercício 1. Portanto,
    pelo exercício 4, queremos mostrar que
    \begin{equation*}
      \sigma_1 = \max_{\substack{y \in \mathbb{R}^m \\ x \in \mathbb{R}^n}}
      \frac{y^TAx}{\norm{x}_2\norm{y}_2}.
    \end{equation*}

    Suponha que a matriz $A$ tenha posto maior ou igual a 1, então $\sigma_1 \neq 0$.

    Agora, considere $x$ como o autovetor unitário associado a $\lambda_1$ da matriz $A^TA$.
    Seja agora $y = Ax/\sigma_1$, onde $\sigma_1 = \sqrt{\lambda_1}$. Então
    \begin{equation*}
      \norm{y}_2^2 = y^Ty = \frac{1}{\sigma_1^2} x^TA^TAx = 1.
    \end{equation*}

    Logo,

    \begin{equation*}
      y^TAx = \frac{x^TA^TAx}{\sigma_1} = \sigma_1.
    \end{equation*}

    Acabamos de mostrar que
    \begin{equation*}
      \max_{\substack{y \in \mathbb{R}^m \\ x \in \mathbb{R}^n}}
      \frac{y^TAx}{\norm{x}_2\norm{y}_2} \geq \sigma_1.
    \end{equation*}

    Vamos mostrar que vale a outra desigualdade agora. Seja $r = \posto(A)$,
    então
    \begin{equation*}
      A = \sum_{i=1}^{r}\sigma_iu_iv_i^T.
    \end{equation*}
    Como os vetores $v_1, \dots, v_n$ formam uma base de $\mathbb{R}^{n}$ e $u_1, \dots, u_m$ de $\mathbb{R}^m$,
    para todo $x \in \mathbb{R}^n$ e $y \in \mathbb{R}^m$,
    \begin{equation*}
      x = d_1 v_1 + \dots d_n v_v, \quad y = c_1 u_1 + \dots + c_m u_m.
    \end{equation*}

    Logo,
    \begin{equation}\label{eq:eqdes}
      \begin{split}
        \dfrac{y^TAx}{\norm{y}_2\norm{x}_2} &= \dfrac{\left( \sum_{i=1}^{m} c_i u_i\right)^T
        \left( \sum_{i=1}^{r} \sigma_iu_iv_i^T \right)
        \left( \sum_{i=1}^{n} d_i v_i\right)}{\left( \sum_{i=1}^{m}c_i^2 \right)^{\frac{1}{2}}
        \left(\sum_{i=1}^{n} d_i^2 \right)^{\frac{1}{2}} } \\
        &= \dfrac{\left( \sum_{i=1}^{m} c_i u_i \right)^T
        \left( \sum_{i=1}^{r} \sigma_i u_i d_i \right)}{\left( \sum_{i=1}^{m}c_i^2 \right)^{\frac{1}{2}}
        \left(\sum_{i=1}^{n} d_i^2 \right)^{\frac{1}{2}} } \\
        &= \dfrac{\sum_{i=1}^{r} \sigma_i c_i d_i}{\left( \sum_{i=1}^{m}c_i^2 \right)^{\frac{1}{2}}
        \left(\sum_{i=1}^{n} d_i^2 \right)^{\frac{1}{2}} } \\
        & \leq \sigma_1 \dfrac{\sum_{i=1}^{r} c_i d_i}{ \left( \sum_{i=1}^{m} c_i^2 \right)^{\frac{1}{2}}
        \left(\sum_{i=1}^{n} d_i^2 \right)^{\frac{1}{2}} }
      \end{split}
    \end{equation}

    Como $r \leq m,n$, considere os vetores
    \begin{equation*}
      c' = (c_1, \dots, c_r), \quad d' = (d_1, \dots, d_r).
    \end{equation*}

    Além disso, considere $c = (c_1, \dots, c_m)$ e $d = (d_1, \dots, d_n)$. Então
    \begin{equation*}
      \sum{i=1}^{r} c_i^2 \leq \sum_{i=1}^{m} c_i^2 \Rightarrow
      \dfrac{1}{\left( \sum{i=1}^{r} c_i^2 \right)^\frac{1}{2}} \geq
      \dfrac{1}{\left( \sum{i=1}^{m} c_i^2 \right)^\frac{1}{2}}.
    \end{equation*}
    De modo análogo para $d$ e $d'$
    \begin{equation*}
      \dfrac{1}{\left( \sum{i=1}^{r} d_i^2 \right)^\frac{1}{2}} \geq
      \dfrac{1}{\left( \sum{i=1}^{n} d_i^2 \right)^\frac{1}{2}}.
    \end{equation*}

    E pela desigualdade de Cauchy-Schwarz, $|\langle c,d \rangle| \leq \norm{c}_2 \norm{d}_2$.
    Portanto,
    \begin{equation*}
      \frac{|\langle c,d \rangle|}{\norm{c}_2 \norm{d}_2} \leq 1.
    \end{equation*}

    Substituindos os valores e desigualdades acima na Equação \eqref{eq:eqdes},
    temos que
    \begin{equation*}
      \sigma_1 \dfrac{\sum_{i=1}^{r} c_i d_i}{ \left( \sum_{i=1}^{m} c_i^2 \right)^{\frac{1}{2}}
      \left(\sum_{i=1}^{n} d_i^2 \right)^{\frac{1}{2}} } \leq
      \sigma_1 \dfrac{\sum_{i=1}^{r} c_i d_i}{ \left( \sum_{i=1}^{r} c_i^2 \right)^{\frac{1}{2}}
      \left(\sum_{i=1}^{r} d_i^2 \right)^{\frac{1}{2}} } \leq \sigma_1.
    \end{equation*}

    Como isto vale para qualqer $x$ e $y$,
    \begin{equation*}
      \sigma_1 = \sigma_{\max}(A) = \max_{\substack{y \in \mathbb{R}^m \\ x \in \mathbb{R}^n}}
      \frac{y^TAx}{\norm{x}_2\norm{y}_2}.
    \end{equation*}
  \end{solution}

  \begin{solution}{(Exercício 7)}

    Vamos usar SVD para provar. Seja $A \in \mathbb{R}^{m\times n}$ uma matriz
    qualquer, e considere a SVD de $A$ dada no exercício 1. Suponha que
    $posto(A) = r$, sendo que $r \leq \min{m,n}$. Suponha que $n \leq m$, a
    demonstração a seguir pode ser usada analogamente para o caso $n > m$.

    Temos
    \begin{equation*}
      A = \sum_{i=1}^{r}\sigma_i u_iv_i^T.
    \end{equation*}

    Seja agora a sequência de matrizes $B_k$, onde
    \begin{equation*}
      B_k = \sum_{i=1}^{n} \left(\sigma_i + \frac{1}{k}\right)u_iv_i^T.
    \end{equation*}

    Observe que as matrizes $B_k$ tem posto completo, já que
    $B_k = U\Sigma_kV^T$, onde $U,V$ são matrizes ortogonais e $\Sigma_k$ possui
    posto cheio, já que as colunas de $\Sigma_k$ são linearmente independentes.

    \begin{equation*}
      \Sigma_k =
      \begin{pmatrix}
        \sigma_1 + \frac{1}{k} & 0 & \cdots & 0 \\
        0 & \sigma_2 + \frac{1}{k} & \cdots & 0 \\
        \vdots & \vdots & \ddots & \vdots \\
        0 & 0 & \cdots & \sigma_n + \frac{1}{k} \\
        0 & 0 & \cdots & 0 \\
        0 & 0 & \cdots & 0
      \end{pmatrix}.
    \end{equation*}

    Além disso, note que $A - B_k = U(\Sigma - \Sigma_k)V^T$, e portanto,
    \begin{equation*}
      \norm{A-B_k}_2 = \norm{U(\Sigma - \Sigma_k)V^T}_2 = \norm{\Sigma - \Sigma_k},
    \end{equation*}

    já que $U$ e $V$ são matrizes ortogonais. Além disso, note que
    \begin{equation*}
      \Sigma - \Sigma_k =
      \begin{pmatrix}
        \sigma_1 & 0 & \cdots & 0 \\
        0 & \sigma_2 & \cdots & 0 \\
        \vdots & \vdots & \ddots & \vdots \\
        0 & 0 & \cdots & \sigma_n \\
        0 & 0 & \cdots & 0 \\
        0 & 0 & \cdots & 0
      \end{pmatrix}
      -
      \begin{pmatrix}
        \sigma_1 + \frac{1}{k} & 0 & \cdots & 0 \\
        0 & \sigma_2 + \frac{1}{k} & \cdots & 0 \\
        \vdots & \vdots & \ddots & \vdots \\
        0 & 0 & \cdots & \sigma_n + \frac{1}{k} \\
        0 & 0 & \cdots & 0 \\
        0 & 0 & \cdots & 0
      \end{pmatrix}
      =
      \begin{pmatrix}
        \frac{1}{k} & 0 & \cdots & 0 \\
        0 & \frac{1}{k} & \cdots & 0 \\
        \vdots & \vdots & \ddots & \vdots \\
        0 & 0 & \cdots & \frac{1}{k} \\
        0 & 0 & \cdots & 0 \\
        0 & 0 & \cdots & 0
      \end{pmatrix}.
    \end{equation*}

    Sabemos que o valor singular $\sigma_1$ de uma matrix $A$ é a raíz quadrada
    do maior autovalor da matriz $A^TA$. Neste caso, o valor singular $\sigma_1$ da
    matriz $\Sigma - \Sigma_k$ é $\frac{1}{k}$ já que o maior autovalor da
    matriz $(\Sigma - \Sigma_k)^T(\Sigma - \Sigma_k)$ é $\frac{1}{k^2}$.
    Pelo exercício 4-b, temos que
    \begin{equation*}
      \norm{\Sigma - \Sigma_k}_2 = \sigma_1.
    \end{equation*}
    Portanto,
    \begin{equation*}
      \norm{A - B_k}_2 = \norm{\Sigma - \Sigma_k}_2 = \sigma_1 = \frac{1}{k}.
    \end{equation*}

    Ou seja, para todo $\epsilon > 0$, existe $k_0$ tal que para todo $k > k_0$,

    \begin{equation*}
      \norm{A - B_k}_2 < \epsilon.
    \end{equation*}

    Assim, mostramos que qualquer matriz em $\mathbb{R}^{m\times n}$ é o limite de uma
    sequência de matrizes de posto completo.
  \end{solution}

  \begin{solution}{(Exercício 8)}

    Seja $\Omega = \{x_1,\dots,x_m\}\subset \mathbb{R}^n$ um conjunto de dados e
    $k\in(1,m)$ um número natural. O Algoritmo de Lloyd ($k$-médias) para clusterização
    consiste em achar centros, ou pontos, de modo que o conjunto de treinamento
    possa ser separado em $k$ partições, todas disjuntas.
    Ou seja, o objetivo é encontrar uma partição ótima
    \begin{equation*}
      S = \{S_1, \dots, S_m\}
    \end{equation*}
    de $\Omega$, isto é,  encontrar $S_i$, chamados de \emph{clusters}, tais que
    \begin{equation*}
      \begin{split}
        & S_i \cap S_j = \emptyset, \quad i\neq j \\
        & \bigcup_{i=1}^k S_i = \Omega.
      \end{split}
    \end{equation*}

    Em termos de minimização, queremos resolver o seguinte problema de minização.

    \begin{equation*}
      \min_{\substack{S \textnormal{ partição} \\ \textnormal{de } \Omega}}
      \sum_{i=1}^{k} \sum_{x_j \in S_i} \norm{x_j - \mu_i}_2^2,
      \quad \textnormal{onde }
      \mu_i = \frac{1}{|S_i|} \sum_{x_j \in S_i} x_j.
    \end{equation*}

    Então, o Algoritmo de Lloyd é dado a seguir.

    \begin{algorithm}[htb]
        \caption{Algoritmo de Lloyd}
        \begin{algorithmic}[1]
            \State Escolha $k$ pontos $m_1^{(1)}, \dots, m_k^{(1)}$ em $\Omega$, os quais
                    serão usados como "médias iniciais". Defina $t=1$.

            \State Para cada $j = 1, \dots, m$, encontre
                  \begin{equation*}
                      i \in  \underset{l=1,\dots,k}{\argmin} \norm{x_j - m_l^{(t)}}
                  \end{equation*}
                  e associe $x_j$ ao cluster $S_i$.
            \State Se $t>1$ e $S^{t} = S^{t-1}$, pare.
            \State (Atualização das médias) Para cada $i=1,\dots,k$, defina
                  \begin{equation*}
                    m_i^{t+1} = \frac{1}{|S_i|} \sum_{x_j \in S_i^{(t)}} x_j
                  \end{equation*}
            \State Defina $t=t+1$ e volte para o passo $2$.
        \end{algorithmic}
    \end{algorithm}

  \end{solution}

  \begin{solution}{(Exercício 9)}

    Seja um banco de dados $\{I_1, \dots, I_m\}$, onde cada $I_i$ é uma
    matriz de imagens em escala cinza. Temos $m$ imagens, de tamanho
    $n\times n$, dividas em $p$ classes, onde cada classe representa a face de uma pessoa.
    Para cada matriz $I_j$, denote por $\Phi_j$ o vetor de concatenação
    das matrizes. Seja $A$ a matriz abaixo.

    \begin{equation*}
      A =
      \begin{pmatrix}
        \vert  & \vert  &        & \vert  \\
        \Phi_1 & \Phi_2 & \cdots & \Phi_m \\
        \vert  & \vert  &        & \vert  \\
      \end{pmatrix}
      \in \mathbb{R}^{n^2 \times m}.
    \end{equation*}

    Para classificar uma nova imagem $I_{new} = \Phi_{new}$, calculamos a
    projeção ortogonal no espaço gerado pelas colunas da matriz $A$, assim
    podemos dizer de qual vetor coluna da matriz $A$, a imagem $\Phi_{new}$
    está mais perto, e esta será a sua classe.
    Portanto, o procedimento é dado no Algoritmo abaixo.

    \begin{algorithm}[htb]
        \caption{Auto-faces para reconhecimendo facial}
        \begin{algorithmic}[1]
            \State Calcule a projeção ortogonal $\hat{\Phi}_{new}$ de $\Phi_{new}$
                    sobre o subespaço gerado por $\Phi_1, \dots, \Phi_m$.
                    \begin{equation*}
                      \hat{\Phi}_{new} = A(A^TA)^{-1}\Phi_{new}
                    \end{equation*}
            \State Calcule
                    \begin{equation*}
                      j = \underset{i=1,\dots,m}{\argmin} \norm{\hat{\Phi}_{new} -
                      \Phi_i}
                    \end{equation*}
            \State Dizemos então que a face em $I_{new}$ é a face da pessoa em
            $I_j$.
        \end{algorithmic}
    \end{algorithm}

  \end{solution}

\end{document}
